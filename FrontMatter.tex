
%\frontmatter % Uncomment to use roman page numbering style (i, ii, iii, iv...) for the pre-content pages

\pagestyle{plain} % Default to the plain heading style until the thesis style is called for the body content


%-------------------------------------------------------------------------------------------------------------------------------
%	Pre-Thesis Content
%-------------------------------------------------------------------------------------------------------------------------------

%----------------------------------------------------------------------------------------
%	TITLE PAGE
%----------------------------------------------------------------------------------------
%
% All requirements of the graduate college as of 2019-01-30
% The title page must be the first page of your document (All pages must be numbered and match 
% the numbers listed in Table of Contents. However, a page number is not required to be printed
% on the actual title page).
% The title page must meet the following requirements:
% Title is set in ALL CAPS
% Student name matches official name in UAccess
% Rule line appears
% Official Department Name is Used
% Degree is indicated correctly
% Copyright year matches year of graduation on page

\begin{titlepage}
\begin{singlespacing} % needed for documents set to 1.5 or 2.0 spacing, Comment out otherwise
\begin{center}

\vfill

\MakeUppercase{\ttitle}\\ %Thesis Title in ALL CAPS
\vspace{0.2in}
by\\ \vspace{0.2in}
{\authorname}\\ % Places author name as specified in preamble
\vspace{0.4in}
\HRule \\[0.1cm] % Horizontal line
Copyright \textcopyright\space\authorname\space{\the\year}\\ % Copyright Date

\vspace{0.2in}

A Dissertation Submitted to the Faculty of the\\ % University required text
\vspace{0.2in}
\MakeUppercase{\deptname} \\  % Department name in Small Caps
\vspace{0.2in}
In Partial Fulfillment of the Requirements \\ \medskip % University required text
For the Degree of \\  % University required text
\vspace{0.2in}
\MakeUppercase{\degreename} \\ % Thesis type
\vspace{0.2in} 
In the Graduate College \\  % University required text
\vspace{0.2in}
\MakeUppercase{The \univname} \\ % University name in Small Caps
\vspace{0.2in}
%\normalsize
{\the\year}\\[4cm] % date
%\includegraphics{Logo} % University/department logo - uncomment to place it

\vfill
\end{center}
\end{singlespacing}% needed for documents set to 1.5 or 2.0 spacing, Comment out otherwise
\end{titlepage}

%\cleardoublepage %Uncomment to add blank page after Title page.


\setcounter{page}{2} % Starts pagination at 2 on the Committee Approval Form with no page number displayed on Title page.

%----------------------------------------------------------------------------------------
%	COMMITTEE APPROVAL PAGE
%----------------------------------------------------------------------------------------
%
% All requirements of the graduate college as of 2019-01-30
% The committee approval page must be the second page of your document
% The committee approval page must meet the following requirements:
% Title on approval page matches title on page 1 (Title Page)
% Dissertation chair (or co-chair) is indicated
% All members and chair (or co-chairs) have signed the approval page
% Date of defense is listed

%\addchaptertocentry{Committee Approval Page} % Add the committee approval page to the table of contents
\begin{singlespacing} % needed for documents set to 1.5 or 2.0 spacing, Comment out otherwise
\begin{center}
%\large

THE \MakeUppercase{\univname} \\
GRADUATE COLLEGE
\end{center}

\vspace*{0.3in}

\noindent As members of the Dissertation Committee, we certify that we have read the dissertation prepared by \authorname \space entitled "\ttitle "\space and recommend that it be accepted as fulfilling the dissertation requirement for the Degree of \degreename.

\vspace*{0.3in}

\noindent\underline{\makebox[4.0in][r]{}} \hspace{0.4in} Date: \defensedate \\
{\bfseries\chairname}\\
\emph{(Chair)} %This line can be commented out if necessary
\vspace*{0.3in}

\noindent\underline{\makebox[4.0in][r]{}} \hspace{0.4in} Date: \defensedate \\
{\bfseries\facnameA}\\
\emph{(Member)} %This line can be commented out if necessary
\vspace*{0.3in}

\noindent\underline{\makebox[4.0in][r]{}} \hspace{0.4in} Date: \defensedate \\
{\bfseries\facnameB}\\
\emph{(Member)} %This line can be commented out if necessary
\vspace*{0.3in}

\noindent\underline{\makebox[4.0in][r]{}} \hspace{0.4in} Date: \defensedate \\
{\bfseries\facnameC}\\
\emph{(Member)} %This line can be commented out if necessary
\vspace*{0.5in}

% If 4th committee member is needed, copy the preceding 4 lines, change to facnameD in copied lines
% You will then need to adjust vertical spacing to keep committee approval page to 1 page length

\noindent Final approval and acceptance of this dissertation is contingent upon the candidate's submission of the final copies of the dissertation to the Graduate College.

\vspace*{0.2in}

\noindent I hereby certify that I have read this dissertation prepared under my direction and recommend that it be accepted as fulfilling the dissertation requirement.
\vspace*{0.5in}

\noindent\underline{\makebox[4.0in][r]{}} \hspace{0.4in} Date: \defensedate \\
Dissertation Director: \chairname \\
%{\bfseries \emph{Instructor \\ Hispanic Linguistics}} % Update hard-coded to job title and department
\vfill
\end{singlespacing}% needed for documents set to 1.5 or 2.0 spacing, Comment out otherwise



%----------------------------------------------------------------------------------------
%	STATEMENT BY AUTHOR
%----------------------------------------------------------------------------------------
%
% No longer required for the graduate college as of 2019-01-30
% Uncomment all lines in this section  with "%%" at the beginning if your document requires it

%%\begin{statement}
%%\begin{singlespacing} % needed for documents set to 1.5 or 2.0 spacing, Comment out otherwise
%%\addchaptertocentry{\authorshipname} % Add the declaration to the table of contents

%The following block of text was the required text of the Graduate College (2019-02-01)
%%This dissertation has been submitted in partial fulfillment of the requirements for an advanced degree at the \univname\space and is deposited in the University Library to be made available to borrowers under rules of the Library. \\ \smallskip 

%%Brief quotations from this dissertation are allowable without special permission, provided that an accurate acknowledgement of the source is made. Requests for permission for extended quotation from or reproduction of this manuscript in whole or in part may be granted by the copyright holder.

%%\vspace*{0.3in}
%%\begin{center} 
%%SIGNED: \authorname
%%\end{center}
%%\end{singlespacing}% needed for documents set to 1.5 or 2.0 spacing, Comment out otherwise
%%\end{statement}


%----------------------------------------------------------------------------------------
%	ACKNOWLEDGEMENTS
%----------------------------------------------------------------------------------------
%
\pagebreak
% Acknowledgements are not a necessary item. Comment out if not being used 
\addchaptertocentry{\acknowledgementname} % Add the acknowledgements to the table of contents

\begin{center}
\MakeUppercase{Acknowledgements}\\ \bigskip
\end{center}
% Nicco
% Don & Lorann
% Rosie and Frank
% Drew
% Atlas
% Sedona
% Samuel
% Lexy
% Jalen
% Manda and Razia
% DLP
% GPSC
% UCW
% TFS
% Campus Pantry
% Forest Defenders
% Michelle, Bronis, Tom
% Committee
% Nicki, Josh, Berlin, Vignesh
% Students of CSC 296
% Marx
% Unnamed computers whose labor is immortalized in our computers, our compilers, and our programs

This document allows you to type your acknowledgements to people, groups and organizations that have helped you along the way\ldots

%----------------------------------------------------------------------------------------
%	DEDICATION
%----------------------------------------------------------------------------------------
%
% Dedications are not a necessary item. Comment out to remove from document
\dedicatory{For the Earth, who nurtures us. \\\bigskip For her stewards, who safeguard our future.} 


%----------------------------------------------------------------------------------------
%	Quotation
%----------------------------------------------------------------------------------------
%
% This page is not really necessary, but if you feel the need to include some quote here is your
% chance. 
%
% Comment out if not being used 
%\include{FrontBackMatter/Quotation} % This calls the Quotation.tex


%----------------------------------------------------------------------------------------
%	LIST OF CONTENTS/FIGURES/TABLES PAGES
%----------------------------------------------------------------------------------------
%
% Table of Contents (TOC) must include:
% a: all major sections with the document in a consistent manner
% b: section headings in document must match their listings (exact words) in TOC
%
\tableofcontents % Prints the main table of contents
%
% Lists of figures and tables must include accurate page numbers
\listoffigures % Prints the list of figures

\listoftables % Prints the list of tables



%----------------------------------------------------------------------------------------
%	Abstract
%----------------------------------------------------------------------------------------
% This is required to appear before the first chapter of the dissertation

\pagebreak

\addchaptertocentry{\abstractname} % Add the abstract to the table of contents

\begin{center}
\MakeUppercase{Abstract}\\ \bigskip
\end{center}


High performance computing is an important tool in various domains, including climate modeling, drug discovery, and recently, generative AI\@.
Application performance and developer productivity are important considerations when developing these codes.
However, optimizing a program's performance for one machine can inadvertently harm its performance for another.
With the ever-growing diversity of computing hardware, it has become more difficult to write code that performs well on multiple different machines, that is, to write code that is \textit{performance portable}.
A similar phenomenon occurs with changes of input data and algorithm choice as well.
Often, developers must maintain multiple versions of their application, one tuned for each system it will be used on.
Performance portability libraries address this problem by separating the description of a computation from its optimization.
Using these libraries, optimizations like tiling, parallelization, and data layout transformations can be quickly and portably applied.
Unfortunately, they leave out important opportunities for optimizations that are applied across multiple loops.
A programmer who wishes to improve data reuse through loop fusion or inter-loop layout changes cannot do so portably.
Similarly, for codes using sparse data structures, the limited available abstractions require code to be format-dependent.
Thus, the performance improvements of changing sparse formats come at the cost of rewriting entire kernels for the new format.
This dissertation remedies these problems, introducing abstractions for cross-loop schedule and data transformations and format-independent abstractions for describing sparse computations.

First, I introduce an interface for cross-kernel scheduling transformations through the RAJALC framework.
The framework uses runtime symbolic evaluation to partially automate and ensure the correctness of the transformations.
On average, this framework requires about a quarter as many code changes while achieving up to 98\% of the performance improvement of a hand-implemented transformation.
Second, I build on the symbolic evaluation capabilities to support as-automated-as-desired format transformations.
Further, I augment RAJA's iteration space capabilities to support triangular iteration spaces.
As with the schedule transformations, performance improvement is achieved with significantly fewer code changes.
Finally, I develop prototype support for the format-independent description of sparse computations.
Because the description of the computation is independent of the sparse format, changing sparse formats becomes as simple as with dense.
The hypothesis was that because the approach to separating the sparse format from the computation description has the same algorithmic complexity as a hand-written sparse loop nest, the performance would be comparable.
However, evaluation shows that there is significant runtime overhead due to the implementation needing to perform checks in the innermost loops and other constant time work.
I conclude that more significant changes to performance portability library interfaces will be needed to support general sparse computations.
Still, the current abstractions are capable of supporting cross-loop scheduling and data transformations for dense codes efficiently. 



\pagestyle{thesis}